\documentclass{report}
\usepackage[utf8]{inputenc}
\usepackage{graphicx}
\usepackage{geometry}
\geometry{top=50pt}

\usepackage{titlesec}
\usepackage{lipsum}
\titleformat{\chapter}[display]
 {\normalfont\bfseries}{}{0pt}{\LARGE} 


\title{SPRINT PLANNING}
\author{Simone Manenti 1048788 \\ Cattaneo Federico 1053265}



\begin{document}

\maketitle
\chapter{Sprint Backlog}
\textbf{Requisiti Must have:}
\begin{itemize}
\item Interfaccia grafica
\item Aggiornamento Grafici in real time
\item Gestione della porta COM tramite interfaccia
\item Gestione dati tramite database locale
\item Export dello storico dati tramite file CSV
\end{itemize}

\textbf{Requisiti Should have:}
\begin{itemize}
\item identificazione tramite colore dei valori che superano la soglia
\end{itemize}

\textbf{Requisiti Could have:}
\begin{itemize}
\item Accesso utenti
\item Possibilità di muovere i componenti grafici all'interno dell'interfaccia
\item Possibilità di cambiare il tema dell'interfaccia
\end{itemize}

\textbf{Requisiti Would have:}
--

\chapter{Sprint Planning 01}
Sprint planning dal 02/01/2024 al 17/01/2024.\\

\textbf{SCOPE}
\begin{itemize}
\item Progettazione interfaccia grafica statica, scelta dei tool, versione beta senza interazione con il database
\item Interfaccia grafica dinamica, aggiornata ciclicamente con i dati più recenti
\item Sviluppo iniziale del database (prototipo)
\end{itemize}


\subsection{Daily sprint 02/01/2024}
\textbf{Fatto:}Abbiamo deciso quale framework utilizzare tra javafx (con l'utilizzo di scenebuilder) o swing (con l'utilizzo di windowbuilder). La scelta migliore per il nostro software è javafx.\\
\textbf{Da fare:} Progettare interfaccia.

\subsection{Daily sprint 04/01/2024}
\textbf{Fatto:} Progettazione dell'interfaccia tramite l'utilizzo di scenebuilder e implementazione del codice.\\
\textbf{Da fare:} gestione delle porte COM e implementazione choisebox per scelta porte COM
\subsection{Daily sprint 05/01/2024}
\textbf{Fatto:} Implementazione porte COM e implementazione scelta porta COM tramite choisebox (GUI)
\textbf{Da fare:} gestione della ricezione dei messaggi

\subsection{Daily sprint 07/01/2024}
\textbf{Fatto:} tentativo di implementazione parsing della stringa in input, da rivedere
\textbf{Da fare:} implementare parsing
\subsection{Daily sprint 08/01/2024}
\textbf{Fatto:} Prototipo implementazione del database locale, inserimento record nel database
\textbf{Da fare:} esportazione dei record dal database
\subsection{Daily sprint 09/01/2024}
\textbf{Fatto:}Scelta della libreria eu.hansolo.medusa per la creazione di indicatori grafici.\\

\subsection{Daily sprint 17/01/2024}
\textbf{Fatto:} Riepilogo sprint settimanale e stesura sprint backlog\\\\

\textbf{Cose fatte durante sptint 01:}
\begin{itemize}
\item Progettazione interfaccia grafica statica, scelta dei tool, versione beta senza interazione con il database
\item Interfaccia grafica dinamica, aggiornata ciclicamente con i dati più recenti, fatta solo in forma tabellare
\item Sviluppo iniziale del database (prototipo)
\end{itemize}

\chapter{Sprint planning 02}
Sprint planning dal 18/01/2024 al 01/02/2024.\\

\textbf{Scope:}
\begin{itemize}
\item completare l'interfaccia grafica con inserimento indicatori grafici
\item Aggiornamento Grafici in real time 
\item terminare gestione dati tramite database locale basandosi sul prototipo
\end{itemize}

\subsection{Daily sprint 18/01/2024}
\textbf{Fatto:}implementazione UI della barra progressi accelleratore e freno\\

\subsection{Daily sprint 19/01/2024}
\textbf{Fatto:}Implementazione parziale UI gauge level, velocità e stato di carica delle batterie\\
Implementazione parziale metodi di aggiornamento dinamico dell'interfaccia.

\subsection{Daily sprint 21/01/2024}
\textbf{Fatto:} Implementazione definitiva dei metodi fatti precedentemente

\subsection{Daily sprint 22/01/2024}
\textbf{Fatto:} Refactoring della gestione delle porte COM\\
Prova per implementazione del lineChart

\subsection{Daily sprint 23/01/2024}
\textbf{Fatto:}Refactoring funzioni del database\\
Prima versione del resize della finestra
\subsection{Daily sprint 24/01/2024}
\textbf{Fatto:}Refactoring della funzione di resize\\
Prima versione del TabPane

\subsection{Daily sprint 28/01/2024}
\textbf{Fatto:} Prima versione del grafico nel tabPane secondario

\subsection{Daily sprint 30/01/2024}
\textbf{Fatto:}  Implementazione dark mode per l'interfaccia grafica

\subsection{Daily sprint 31/01/2024}
\textbf{Fatto:}Caricamento documentazione con aggiornamenti

\subsection{Daily sprint 01/02/2024}
\textbf{Fatto:}Scelta provissoria della versione di javafx da utilizzare a causa dell'implementazione del grafico


\chapter{Sprint planning 03}
Sprint planning dal 19/02/2024 al 03/03/2024.\\

\textbf{Scope:}
\begin{itemize}
\item Capire come implementare il grafico in real time
\item Refactoring delle classi
\item Aggiungere test Junit

\subsection{Daily sprint 19/02/2024}
\textbf{Fatto:} Prima versione del codice con l'aggiunta di una classe nel view

\subsection{Daily sprint 24/02/2024}
\textbf{Fatto:} Versione provvisoria del codice con le classi riorganizzate
\subsection{Daily sprint 25/02/2024}
\textbf{Fatto:} Aggiunta UI per test linechart
\subsection{Daily sprint 26/02/2024}
\textbf{Fatto:} Implementazione provvisoria della logica per il grafico in real time


\end{itemize}

\end{document}

