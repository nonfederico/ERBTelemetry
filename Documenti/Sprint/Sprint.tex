\documentclass{report}
\usepackage[utf8]{inputenc}
\usepackage{graphicx}
\usepackage{geometry}
\geometry{top=50pt}

\usepackage{titlesec}
\usepackage{lipsum}
\titleformat{\chapter}[display]
 {\normalfont\bfseries}{}{0pt}{\LARGE} 


\title{SPRINT PLANNING}
\author{Simone Manenti 1048788 \\ Cattaneo Federico 1053265}



\begin{document}

\maketitle
\chapter{Sprint Backlog}
\textbf{Requisiti Must have:}
\begin{itemize}
\item Interfaccia grafica
\item Aggiornamento Grafici in real time
\item Gestione della porta COM tramite interfaccia
\item Gestione dati tramite database locale
\item Export dello storico dati tramite file CSV
\end{itemize}

\textbf{Requisiti Should have:}
\begin{itemize}
\item identificazione tramite colore dei valori che superano la soglia
\item Accesso utenti
\end{itemize}

\textbf{Requisiti Could have:}
\begin{itemize}
\item Possibilità di muovere i componenti grafici all'interno dell'interfaccia
\item Possibilità di cambiare il tema dell'interfaccia
\end{itemize}

\textbf{Requisiti Would have:}

\chapter{Sprint Planning 01}
Sprint planning dal 02/01/2024 al 09/01/2024.\\

\textbf{SCOPE}
\begin{itemize}
\item Progettazione interfaccia grafica statica, scelta dei tool, versione beta senza interazione con il database
\item Interfaccia grafica dinamica, aggiornata ciclicamente con i dati più recenti
\item Sviluppo iniziale del database (prototipo)
\end{itemize}


\subsection{Daily sprint 02/01/2024}
\textbf{Fatto:}Abbiamo deciso quale framework utilizzare tra javafx (con l'utilizzo di scenebuilder) o swing (con l'utilizzo di windowbuilder). La scelta migliore per il nostro software è javafx.\\
\textbf{Da fare:} Progettare interfaccia.
\subsection{Daily sprint 03/01/2024}
--
\subsection{Daily sprint 04/01/2024}
\textbf{Fatto:} Progettazione dell'interfaccia tramite l'utilizzo di scenebuilder e implementazione del codice.\\
\textbf{Da fare:} gestione delle porte COM e implementazione choisebox per scelta porte COM
\subsection{Daily sprint 05/01/2024}
\textbf{Fatto:} Implementazione porte COM e implementazione scelta porta COM tramite choisebox (GUI)
\textbf{Da fare:} gestione della ricezione dei messaggi
\subsection{Daily sprint 06/01/2024}
--
\subsection{Daily sprint 07/01/2024}
\textbf{Fatto:} tentativo di implementazione parsing della stringa in input, da rivedere
\textbf{Da fare:} implementare parsing
\subsection{Daily sprint 08/01/2024}
\textbf{Fatto:} Prototipo implementazione del database locale, inserimento record nel database
\textbf{Da fare:} esportazione dei record dal database
\subsection{Daily sprint 09/01/2024}
Riepilogo sprint settimanale e stesura sprint backlog\\
Scelta della libreria eu.hansolo.medusa per la creazione di indicatori grafici.\\\\

\textbf{Cose fatte durante sptint 01:}
\begin{itemize}
\item Progettazione interfaccia grafica statica, scelta dei tool, versione beta senza interazione con il database
\item Interfaccia grafica dinamica, aggiornata ciclicamente con i dati più recenti, fatta solo in forma tabellare
\item Sviluppo iniziale del database (prototipo)
\end{itemize}

\chapter{Sprint planning 02}
Sprint planning dal 22/01/2024 al 29/01/2024.\\

\textbf{Scope:}
\begin{itemize}
\item completare l'interfaccia grafica con inserimento indicatori grafici
\item Aggiornamento Grafici in real time 
\item terminare gestione dati tramite database locale basandosi sul prototipo
\end{itemize}

\subsection{Daily sprint 22/01/2024}
\textbf{Fatto:} Modifica interfaccia grafica 
\textbf{Da fare:} implementazione grafici real time e modifica colore variabili in base al valore


\end{document}