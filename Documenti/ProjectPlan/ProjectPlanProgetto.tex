\documentclass{report}
\usepackage[utf8]{inputenc}
\usepackage{graphicx}
\usepackage{geometry}
\geometry{top=50pt}

\usepackage{titlesec}
\usepackage{lipsum}
\titleformat{\chapter}[display]
 {\normalfont\bfseries}{}{0pt}{\LARGE} 


\title{PROJECT PLAN - ERB TELEMETRY}
\author{Simone Manenti 104878\\ Cattaneo Federico 1053265}



\begin{document}

\maketitle

\chapter{INTRODUZIONE}
\subsection{Storia}
Durante la fase di progettazione svolta dal gruppo di controllo del team ERB di formula SAE, dell'università degli studi di Bergamo, si è scoperta la necessità di dover leggere i dati dei sensori collegati alla centralina, visualizzarli a schermo per poter fare un'analisi in tempo reale e verificare che nel sistema tutti i sensori siano in uno stato accettabile.
Non trovando un software gratuito che andasse in contro alle esigenze del team, abbiamo deciso di sviluppare un software con le funzionalità essenziali per poter far procedere il team con la fase di test.
\\
Le necessità del team sono di visualizzare a schermo i dati in tempo reale, tramite tabelle o tramite grafici (funzionalità opzionale) e di salvare questi dati da qualche parte, per poter eventualmente confrontare i dati di due sessioni di test e capire se, a seguito di modifiche alla vettura, queste siano efficaci oppure siano modifiche non necessarie o controproducenti.


\chapter{Descrizione del progetto}
Nel sistema sono presenti due arduino:
\begin{itemize}
\item \textbf{arduinoTX}: collegato alla centralina, si occupa della trasmissione dei dati dalla vettura ad un ricevitore.
\item \textbf{arduinoRX}: collegato ad un computer, si occupa della ricezione dei dati inviati dall'arduinoTX.
\end{itemize}

\subsection{Funzionalità}

\begin{itemize}
\item Riceviamo i dati in real time dei sensori collegati alla centralina nel formato: "accelleratore, pressione freno, temp. freno, temp. batteria, carica batteria, tempo sul giro, pressione gomme".
\item I dati possono essere visualizzati in real time tramite tabelle o grafici.
\item I dati possono essere salvati in un database locale .
\item Possiamo accedere allo storico dei dati salvati nel database locale.
\item L'utente che è autorizzato, può eliminare i dati non necessari, presenti nel database locale.
\end{itemize}


\subsection{Utenti}
Ci sono due possibili Livelli utente, “CapoReparto” e “IngegnereDiPista”.\\

L'IngegnereDiPista può:\\
\begin{itemize}
\item Accedere ai dati della telemetria e visualizzarli a schermo tramite grafici o tabelle.
\item Selezionare quali tipi di dati sono da visulizzare.
\item Selezionare l’intervallo di tempo dei dati che vuole visualizzare.\\
\end{itemize}

Il CapoReparto può:
\begin{itemize}
\item Svolgere tutte le funzioni dell’IngegnereDiPista.
\item Selezionare dei dati ed eliminarli dal database locale.
\end{itemize}

\subsection{Comportamento comunicazione real time}
\begin{itemize}
\item Qualsiasi utente, tramite pulsante, decide quando tentare di avviare la comunicazione tra i due arduino.
Se la comunicazione è avvenuta, i dati possono essere visualizzati tramite tabella e contemporaneamente salvati in locale.
\item L’utente deve anche stoppare la comunicazione tra i due arduino (implica lo stop della lettura e scrittura dei dati).
\item Se la connessione tra arduino non avviene, deve apparire un messaggio di errore.
\end{itemize}

\subsection{Comportamento dati in locale}
\begin{itemize}
\item Qualsiasi utente può visualizzare i dati salvati nel database locale tramite tabelle o grafici.
\item Qualsiasi utente può esportare i grafici o le tabelle.
\item L’utente CapoReparto può selezionare ed eliminare i dati presenti nel Database.
\end{itemize}





\chapter{MODELLO DI PROCESSO}
Abbiamo deciso di adottare un modello di processo Agile ibrido per i seguenti motivi:
\begin{itemize}
\item Il team ERB attualmente non ha una visione completa del sistema
\item Vogliamo procedere a piccoli incrementi per poter avere una qualità maggiore 
\item Potrebbero avvenire cambiamenti dell'interfaccia e delle funzioni
\item Dobbiamo ottimizzare il tempo a disposizione per lo sviluppo del sistema
\end{itemize}
aggiungi tabella scrum con il punto di partenza con tutte le attività da svolgere.\\
aggiungi MoSCoW guardando i seguenti obbiettivi:\\\\
Dobbiamo sviluppare un software in grado di gestire i dati ricevuti.\\
I dati li salveremo in un database locale, in forma tabellare.\\
Dobbiamo avere delle funzioni per poter decidere se visualizzarli tramite tabelle o grafici.\\
Per gestire la modifica dei dati, introduciamo una schermata di login all'apertura del software, così da poter determinare se l'utente ha l'autorizzazione per poter eventualmente, eliminare i dati presenti nel database locale.\\
Dobbiamo avere una funzione che ci permetta di esportare i dati salvati nel database locale.\\
Dobbiamo predisporre dei popup che comunichino all'utente se la comunicazione tra i trasmettitore e ricevitore è andata a buon fine.\\ 


\chapter{ORGANIZZAZIONE DEL PROGETTO}
Avendo deciso di usare un modello SCRUM, ci troviamo a dover assegnare ognuno dei tre ruoli ad entrambi i membri del Team.\\
I ruoli di SCRUM Master, Product owner e Development Team saranno ricoperti da:
\begin{itemize}
\item Cattaneo Federico
\item Simone Manenti
\end{itemize}
Entrambi i membri del Team hanno pari responsabilità all'interno del progetto.

\begingroup
\let\clearpage\relax
\chapter{STANDARD. LINEE GUIDA E PROCEDURE}
\endgroup

Le procedure che il team dovrà seguire sono:
\begin{itemize}
\item Individuazione dei requisiti, stima dei costi e delle tempistiche
\item Progettazione dei modelli UML su cui si baserà il codice
\item Implementazione del codice
\item Verifica della qualità e della validità
\end{itemize}
Per la gestione delle versioni del sistema utiliziamo GIT, Per alcune componenti useremo uno sviluppo simile a quello adottato nell'extreme programming, nel quale entrambi i membri del Team programmeranno sulla stessa macchina.\\
Gestiamo la priorità dei requisiti secondo le regole MoSCoW\\
Adotteremo le linee guida ISO/IEC 9126


\chapter{ATTIVITA' DI GESTIONE}
Dopo esserci occupati dell'individuazione dei requisiti e aver fatto la prima stesura del Product Backlog, dobbiamo occuparci dello Sprint Planning.	\\
\textbf{Sprint:}\\
\begin{itemize}
\item La durata di ogni sprint sarà di 7 giorni.
\item Al termine del primo Sprint avverrà il rilascio di una prima versione del software, che non conterrà tutti i requisiti individuati nel Product Backlog, ma fungerà da Prototipo.
\item Al termine del secondo Sprint, verrà rilasciata una versione software contenente tutti i requisiti "Must Have".
\item Nei successivi Sprint andremo ad implementare i requisiti "Should have" ed eventuali nuovi requisiti "Must Have" e ci concentreremo principalmente sul miglioramento degli elementi già implementati nel software.
\item Al termine di ogni Sprint presentiamo il software al team ERB per avere un feedback utile per apportare modifiche o individuare nuove funzionalità.
\end{itemize}

\textbf{Daily scrum:}\\
\begin{itemize}
\item I membri del team, a fine giornata, si riuniranno per pianificare l'attività del giorno dopo.
\end{itemize}


\chapter{RISCHI}
Abbiamo individuato diverse tipologie di rischi:
\begin{itemize}
\item Ritardo nel rappresentare i dati ricevuti sul TX
\item Possibilità di ottenere grafici non intuitivi da leggere
\item Errore di eliminazione dei dati sul database da parte del CapoReparto
\item Possibilità di perdita dei dati nel caso in cui ci sia un guasto dell'hardware, poichè il database è locale, sulla macchina dell'utente
\end{itemize}

\begingroup
\let\clearpage\relax
\chapter{PERSONALE}
\endgroup
Entrambi i membri del team:
\begin{itemize}
\item Entrambi i membri del team avranno pari responsabilità e si divideranno equamente le attività da svolgere.
\item partecipano, gratuitamente, dalle prime fasi del progetto.
\item hanno dovuto imparare ad utilizzare i vari software e tools proposti per lo sviluppo del sistema, ad esempio:
 Maven, StarUml, GitHub.
\item hanno conoscenze di base del linguaggio Java.
\item Non hanno competenze sulla libreria SQLite.
\end{itemize}

\chapter{METODI E TECNICHE}
Per lo sviluppo del progetto utiliziamo diversi software e tools:
\begin{itemize}
\item Eclipse e Maven
\item SQLite
\item GitHub
\item StarUML
\end{itemize}
Per gestire la configurazione e tener traccia delle modifiche apportate al software, utiliziamo la notazione orientata alla versione;\\
Ogni modifica genererà una versione numerata del software (ad esempio: X.0.1).
\chapter{GARANZIA DI QUALITA'}
DA COMPLETARE
\chapter{PACCHETTI DI LAVORO}
DA COMPLETARE
\chapter{RISORSE}
Le funzioni svolte dal software non richiedono grandi quantità di potenza computazionale e generalmente non richiedono grandi quantità di memoria per il salvataggio dei dati.\\

\begingroup
\let\clearpage\relax
\chapter{BUDGET}
\endgroup

Il progetto è diviso in Milestone:
\textbf{Inizio del progetto:} 
\begin{itemize}
\item L'obiettivo è di individuare il problema da risolvere, i requisiti per risolverlo e descrivere gli obiettivi.
\item Concluso in data 15/12/2023
\end{itemize}

\textbf{Stesura Project Plan:}
\begin{itemize}
\item L'obiettivo è quello di generare il project plan
\item Concluso in data 24/12/2023
\end{itemize}

\textbf{Elaborazione dei modelli UML:}
\begin{itemize}
\item L'obiettivo è quello di ottenere l'architettura del software tramite diagrammi UML
\item Concluso in data 29/12/2023
\end{itemize}

\textbf{Generazione del codice:}
\begin{itemize}
\item L'obiettivo è quello di generare il codice a partire dai modelli UML ottenuti nella Milestone precedente ed avere le prime versioni preliminari del software
\item Da concludere entro 07/01/2024
\end{itemize}

\textbf{Conclusione del progetto:}
\begin{itemize}
\item L'obiettivo è quello di verificare e validare il software e consegnare la versione definitiva al team ERB
\item Da concludere entro 19/01/2024
\end{itemize}

\chapter{CAMBIAMENTI}
Al termine di ogni Sprint, se vengono proposti dei cambiamenti da parte del team ERB, verranno inseriti all'interno del Product Backlog per poter procedere con lo Sprint planning.\\
Verranno anche aggiornati i documenti e verranno rilasciate le versioni del software con la numerazione descritta precedentemente.

\begingroup
\let\clearpage\relax
\chapter{CONSEGNA}
\endgroup

Al termine di ogni sprint, nel caso in cui siano state sviluppate nuove versioni del software, verrà consegnata al team ERB la versione più recente e verranno descritte le funzionalità aggiunte.\\
Ad ogni versione consegnata al team ERB, verrà rilasciata anche la documentazione inerente per poter facilitare l'utilizzo del software.\\\\
La versione finale del software verrà rilasciata solo dopo la fase di validazione e verifica fatta quando tutti i requisiti presenti nella categoria "Must Have" sono presenti all'interno del software.


\end{document}