\documentclass{report}
\usepackage[utf8]{inputenc}
\usepackage{graphicx}
\usepackage{geometry}
\geometry{top=20pt}


\title{PROJECT PLAN - ERB TELEMETRY}
\author{Simone Manenti 104878\\ Cattaneo Federico 1053265}


\begin{document}

\maketitle

\chapter{STORIA}
Il progetto è nato dall’esigenza di avere un software di telemetria per il team di formula SAE, dell’università degli studi di Bergamo, da utilizzare durante i test e la competizione per poter leggere i dati raccolti e operare sulle variabili che influenzano la risposta della centralina

\chapter{DESCRIZIONE DEL PROBLEMA}
Nel sistema sono presenti due arduino, un arduino è collegato alla centralina dell'auto e si occupa di inviare costantemente i dati, un altro arduino invece è collegato al computer e si occupa di ricevere i dati dall'altro arduino.\\

\chapter{Funzionalità}
L'arduino può:\\

\begin{itemize}
\item Ricevere i dati in real time dei sensori collegati alla centralina: "accelleratore, pressione freno, temp. freno, temp. batteria, carica batteria, tempo sul giro, pressione gomme"
\item 2)	Questi dati possono essere visualizzati in real time e salvati in un database locale oppure posso accedere allo storico dei dati salvati nel database locale\\
\end{itemize}

All'apertura dell'applicazione è presente una finestra di login per accedere con nome utente e password.\\

\subsection{UTENTI}
Ci sono due possibili Livelli utente, “CapoReparto” e “IngegnereDiPista”.\\

L'IngegnereDiPista può:\\
\begin{itemize}
\item Accedere ai dati della telemetria e visualizzarli a schermo tramite grafici o tabelle 
\item Selezionare quali tipi di dati sono da visulizzare
\item Selezionare l’intervallo di tempo dei dati che vuole visualizzare\\
\end{itemize}

Il CapoReparto può:
\begin{itemize}
\item Svolgere tutte le funzioni dell’IngegnereDiPista
\item Eliminare i dati presenti nel database locale
\end{itemize}

\subsection{COMPORTAMENTO COMUNICAZIONE REAL TIME}
\begin{itemize}
\item Qualsiasi utente, tramite pulsante, decide quando tentare di avviare la comunicazione tra i due arduino.
Se la comunicazione è avvenuta, i dati possono essere visualizzati tramite tabella e contemporaneamente salvati in locale.
\item L’utente deve anche stoppare la comunicazione tra i due arduino (implica lo stop della lettura e scrittura dei dati)
\item Se la connessione tra arduino non avviene, deve apparire un messaggio di errore.
\end{itemize}

\subsection{COMPORTAMENTO DATI IN LOCALE}
\begin{itemize}
\item Qualsiasi utente può visualizzare i dati tramite grafici
\item Qualsiasi utente può esportare i grafici o le tabelle tramite file csv
\item L’utente CapoReparto può eliminare i dati presenti nel Database
\end{itemize}

\end{document}